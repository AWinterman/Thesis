\chapter{Conclusion}
Two tests of equivalence were described in theory and in application to microarray data analysis. Code to carry out the application of both tests can be found in Appendix \ref{Code}. There is quite a bit left to be done. The greater power of the bootstrap when applied to normal data found in chapter \ref{Analysis} runs counter to common sense. It would seem to imply that it is better to ignore clear evidence that data is normal than to take this information into account (for small sample sizes).

 Possible explanations for the this surprising result are an error in the R-script use to run the tests, an error in the generation of data, or that the common sense the result contradicts is simply wrong. The R function used to generate data is well tested, leaving code error as the remaining flaw. With more time, an alternative execution of both tests should be implemented so as to corroborate results, and the code of both implementations should be carefully inspected. Additionally, perhaps there is a theoretical argument for the superiority of the Bootstrap over the t-test in the one-sided case with normal data. In any event, further research is needed.
 
It would be appropriate to apply these tests to the analysis of data from an experiment designed to produce equivalent data sets. For example, Susan C. P. Renn's laboratory compiled a data set in which every sample on the microarray came from the same source, at roughly the same time. We can reasonably assert that every sample is equivalent to the reference\footnote{If a tissue is not equivalent to itself, then the term has no meaning}, and hence use this data set to assess the power of the two tests of equivalence.

Finally, a better critical interval could be developed. It is not at all clear that the interval $I_\epsilon = (-\epsilon, \epsilon)$ is the best one for these purposes. It is simply the simplest and most intuitive. There may be good reason to use a smaller interval - to reduce the probability of type II error - or a larger one to increase power. 